\documentclass[a4paper,10pt,notitlepage]{article}
\usepackage{ctex,geometry,graphicx,tikz,setspace,paralist,fancyhdr,caption}
\geometry{
	left=1.5cm,
	right=1.5cm,
	top=1.4cm,
	bottom=2cm,}
\newcommand{\rec}{
	\begin{tikzpicture}[remember picture,overlay]
		% 绘制边框
		\draw[line width=1.2pt] ([xshift=0.5cm,yshift=0.5cm] current page.south west) rectangle ([xshift=-0.5cm,yshift=-0.8cm] current page.north east);
	\end{tikzpicture}
	
}
\pagestyle{fancy}
\fancyhf{}
\fancyhead[C]{\rec}  % 在页眉中绘制图形
\fancyhead[L]{附录}
\fancyhead[R]{实验九 \quad 有源滤波器}
\fancyfoot[C]{\thepage}
\begin{document}
	\large
	\onehalfspacing
	\begin{figure}[h]
		\raggedright
		\includegraphics{1.png}
	\end{figure}
	\centering
	{\Huge\textbf{模电实验报告}\par}
	\vspace{0.2cm}
	{\huge{实验内容:有源滤波器}\par}
	\raggedright
	\vspace{0.3cm}
	\begin{centering}
		{\large 院系:电子与信息工程学院\hfill 学号:22309080\hfill 审批:\hspace{2cm} \par
			专业:通信工程\hfill 实验人:梁倍铭\hfill 日期:2023年12月5日 \par}
	\end{centering}
	\vspace{0.3cm}
	\section*{一、低通滤波器}
	通带放大倍数为:
	$$A_{up}=1+\frac{R_F}{R_1}=1+\frac{2k}{10k}=1.2(R_F=2k)$$
	$$A_{up}=1+\frac{R_F}{R_1}=1+\frac{10k}{10k}=2(R_F=10k)$$
	截至频率:
	$$f_0=\frac{1}{2\pi RC}\approx 159$$
	\begin{table}[h]
		\begin{minipage}{0.3\textwidth}
			\centering
			\begin{tabular}{|c|c|c|c|c|c|c|c|c|c|c|c|c|c|}
				\hline
				频率(Hz) & 4 & 10 & 20 & 30 & 50 & 100 & 150 & 180 & 200 & 300 & 400 & 600 & 800 \\
				\hline
				仿真$V_o$(V) & 1.195 & 1.185 & 1.15 & 1.095 & 0.93 & 0.51 & 0.28 & 0.2055 & 0.171 & 0.08 & 0.0465 & 0.02075 & 0.0117 \\
				\hline 
				实验$V_o$(V) & \qquad & \qquad & \qquad & \qquad & \qquad & \qquad & \qquad & \qquad & \qquad & \qquad & \qquad & \qquad &\qquad \\
				\hline
			\end{tabular}
			\caption*{表1 低通滤波器$R_F$取2k时的$V_o$值}
		\end{minipage}
		\\
		\begin{minipage}{0.3\textwidth}
			\centering
			\begin{tabular}{|c|c|c|c|c|c|c|c|c|c|c|c|c|c|}
				\hline
				频率(Hz) & 4 & 10 & 20 & 30 & 50 & 100 & 150 & 180 & 200 & 300 & 400 & 600 & 800 \\
				\hline
				仿真$V_o$(V) & 2 & 2.015 & 2.06 & 2.125 & 2.28 & 1.43 & 0.625 & 0.422 & 0.337 & 0.1445 & 0.08 & 0.035 & 0.0197 \\
				\hline 
				实验$V_o$(V) & \qquad & \qquad & \qquad & \qquad & \qquad & \qquad & \qquad & \qquad & \qquad & \qquad & \qquad & \qquad & \qquad \\
				\hline
			\end{tabular}
			\caption*{表2 低通滤波器$R_F$取10k时的$V_o$值}
		\end{minipage}
	\end{table}
	\section*{二、高通滤波器}
	通带放大倍数为:
	$$A_{up}=1+\frac{R_F}{R_1}=1+\frac{2k}{10k}=1.2(R_F=2k)$$
	$$A_{up}=1+\frac{R_F}{R_1}=1+\frac{10k}{10k}=2(R_F=10k)$$
	截至频率:
	$$f_0=\frac{1}{2\pi RC}\approx 159$$
	\begin{table}[h]
		\begin{minipage}{0.3\textwidth}
			\centering
			\begin{tabular}{|c|c|c|c|c|c|c|c|c|c|c|c|c|}
				\hline
				频率(Hz) & 400 & 350 & 300 & 250 & 200 & 180 & 160 & 140 & 100 & 50 & 20 & 10 \\
				\hline
				仿真$V_o$(V) & 1.17 & 1.16 & 1.15 & 1.125 & 1.08 & 1.06 & 1.025 & 0.98 & 0.815 & 0.371 & 0.073 & 0.01885\\
				\hline 
				实验$V_o$(V) & \qquad & \qquad & \qquad & \qquad & \qquad & \qquad & \qquad & \qquad & \qquad & \qquad & \qquad & \qquad \\
				\hline
			\end{tabular}
			\caption*{表3 高通滤波器$R_F$取2k时的$V_o$值}
		\end{minipage}
		\\
		\begin{minipage}{0.3\textwidth}
			\centering
			\begin{tabular}{|c|c|c|c|c|c|c|c|c|c|c|c|c|}
				\hline
				频率(Hz) & 400 & 350 & 300 & 250 & 200 & 180 & 160 & 140 & 100 & 50 & 20 & 10 \\
				\hline
				仿真$V_o$(V) & 2.035 & 2.05 & 2.065 & 2.095 & 2.145 & 2.17 & 2.21 & 2.25 & 2.28 & 0.91 & 0.131 & 0.032 \\
				\hline 
				实验$V_o$(V) & \qquad & \qquad & \qquad & \qquad & \qquad & \qquad & \qquad & \qquad & \qquad & \qquad & \qquad & \qquad \\
				\hline
			\end{tabular}
			\caption*{表4 高通滤波器$R_F$取10k时的$V_o$值}
		\end{minipage}
	\end{table}
	\section*{三、带通滤波器}
	中心频率为:
	$$f_0=\frac{1}{2\pi RC}\approx 159$$
	放大倍数:
	$$A_{up}=1+\frac{R_F}{R_1}=1+\frac{10k}{10k}=2$$
	\begin{table}[h]
		\centering
		\begin{tabular}{|c|c|c|c|c|c|c|c|c|c|c|c|c|c|}
			\hline
			频率(Hz) & 10 & 20 & 50 & 100 & 120 & 150 & 160 & 180 & 200 & 250 & 300 & 350 & 400 \\
			\hline
			仿真$V_o$(V) & 0.126 & 0.253 & 0.66 & 1.44 & 1.735 & 1.985 & 2 & 1.94 & 1.805 & 1.45 & 1.175 & 0.99 & 0.85 \\
			\hline 
			实验$V_o$(V) & \qquad & \qquad & \qquad & \qquad & \qquad & \qquad & \qquad & \qquad & \qquad & \qquad & \qquad & \qquad & \qquad \\
			\hline
		\end{tabular}
		\caption*{表5 带通滤波器$V_o$值}
	\end{table}
\end{document}